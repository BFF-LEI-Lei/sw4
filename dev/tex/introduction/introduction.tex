\documentclass[fleqn]{article}
\usepackage[left=1in, right=1in, top=1in, bottom=1in]{geometry}
\usepackage{mathexam}
\usepackage{amsmath}
\usepackage{tikz}
\usepackage{bm}

%\ExamClass{Sample Class}
%\ExamName{Sample Exam}
%\ExamHead{\today}

\let\ds\displaystyle

\begin{document}
In this project, we use the fourth order summation-by-parts method to solve the three dimensional elastic wave equation in isotropic materials on curvilinear grids. For the physical domain, we consider Dirichlet boundary conditions for both $x$ and $y$ directions and they are also Cartesian grids, but for the z direction, it has different top surface geometry (free surface condition is considered), bottom surface geometry (Dirichlet boundary condition is considered), and interface geometry (you can decide the position of the interface). We divide the physical domain to be fine domain and coarse domain based on the location of the interface, then above the interface, the geometry is the linear interpolation of the top surface geometry and the interface geometry; below the interface, the geometry is the linear interpolation of the interface geometry and the bottom surface geometry. We use two different fourth order summation-by-parts finite difference methods in coarse domain and fine domain. For the coarse domain, we use ghost points to approximate the second derivative in $z$ direction, but for the fine domain,  we don’t use the ghost points to approximate the second derivative in $z$ direction, thus on the mesh refinement interface, we only need to solve linear system which considers the ghost points for coarse domain to be unknowns. Considering this is a three dimensional problem, we also develop iterative methods to solve the linear system, specifically, we use block Jacobina, conjugate gradient and preconditioned conjugate gradient methods. In our numerical experiments, we find that the preconditioned conjugate gradient method works best, it converges around $9$ iterations, and block Jacobian converges around $13$ iterations, conjugate gradient method converges around $44$ iterations. 

\end{document}