%!TEX root = elastic_3d_sbp.tex
\subsubsection{Energy estimate}
From the definitions of matrcies $N_{ij}$, we can easily verify that $N_{11}, N_{22}, N_{33}$ are positive definite and $N_{ij} = N_{ji}^T$. In the curvilinear coordinates, the $L^2$ scalar product is scaled by $J$. We define the $L^2$ scalar product of the two real vector-valued functions ${\bf u}({\bf x})\in\mathbb{R}^3\rightarrow \mathbb{R}^3$ and ${\bf v}(\bf x)\in\mathbb{R}^3\rightarrow \mathbb{R}^3$ by
\begin{equation*}
({\bf u},{\bf v})_2 = \int_{\Omega}  \left(\sum_{l=1}^3 u^{l}v^{l}\right)dx^{(1)}dx^{(2)}dx^{(3)} = \int_{{\bf r}\in[0,1]^3} \left(\sum_{l=1}^3 u^{l}v^{l}\right)Jdr^{(1)}dr^{(2)}dr^{(3)}.
\end{equation*}
To analyze the energy estimate of the solution of an elastic wave equation in the curvilinear domain, we multiply the equation (\ref{elastic_eq_curvi}) by $J{\bf u}_t$ and integrate over the parameter space $[0,1]^3$ to obtain
\begin{align}\label{time_deri_curvi}
({\bf u}_t,\rho {\bf u}_{tt})_2 &= \left({\bf u}_t,\frac{1}{J}\bar{\partial}_1(\bar{A}_1\bar{\nabla}{\bf u})\right)_2+ \left({\bf u}_t,\frac{1}{J}\bar{\partial}_2(\bar{A}_2\bar{\nabla}{\bf u})\right)_2+ \left({\bf u}_t,\frac{1}{J}\bar{\partial}_3(\bar{A}_3\bar{\nabla}{\bf u})\right)_2+({\bf u}_t, {\bf F})_2\nonumber\\
&:= -S({\bf u}_t,{\bf u}) + B({\bf u}_t,{\bf u}) + ({\bf u}_t,{\bf F})_2,
\end{align}
where the term $S$ represents the interior terms after integration by parts,
\begin{multline*}
S({\bf u}_t,{\bf u}) = \int_{{\bf r}\in[0,1]^3} (\bar{\partial}_1 {\bf u}_t)^T ({\bar A}_1\bar{\nabla}{\bf u}) + (\bar{\partial}_2 {\bf u}_t)^T ({\bar A}_2\bar{\nabla}{\bf u}) + (\bar{\partial}_3 {\bf u}_t)^T ({\bar A}_3\bar{\nabla}{\bf u})\ dr^{(1)}dr^{(2)}dr^{(3)}\\
= \int_{{\bf r}\in[0,1]^3} (\bar{\partial}_1 {\bf u}_t)^T (N_{11}\bar{\partial}_1{\bf u} +N_{12}\bar{\partial}_2{\bf u}+ N_{13}\bar{\partial}_3{\bf u}) + (\bar{\partial}_2 {\bf u}_t)^T (N_{21}\bar{\partial}_1{\bf u} +N_{22}\bar{\partial}_2{\bf u}+ N_{23}\bar{\partial}_3{\bf u}) \\
+(\bar{\partial}_3 {\bf u}_t)^T (N_{31}\bar{\partial}_1{\bf u} +N_{32}\bar{\partial}_2{\bf u}+ N_{33}\bar{\partial}_3{\bf u})\ dr^{(1)}dr^{(2)}dr^{(3)}\\
= \sum_{i = 1}^3\sum_{j=1}^3  \int_{{\bf r}\in[0,1]^3} (\bar{\partial}_i {\bf u}_t)^T(N_{ij}\bar{\partial}_j {\bf u})\ dr^{(1)}dr^{(2)}dr^{(3)}.
\end{multline*}
Recall the definition of $N_{ij}, i,j = 1,2,3$ in (\ref{definition_Nij}), we have
\begin{multline*}
S({\bf u}_t,{\bf u}) = \sum_{i=1}^3\sum_{j=1}^3 \int_{{\bf r}\in[0,1]^3} (\bar{\partial}_i {\bf u}_t)^T(J\bar{P}_i^TC\bar{P}_j\bar{\partial}_j {\bf u})\ dr^{(1)}dr^{(2)}dr^{(3)} \\
= \sum_{i=1}^3\sum_{j=1}^3 \int_{{\bf r}\in[0,1]^3} (\bar{P}_i\bar{\partial}_i {\bf u}_t)^TC(\bar{P}_j\bar{\partial}_j {\bf u}) J\ dr^{(1)}dr^{(2)}dr^{(3)}.
\end{multline*}
Since $C$ is a symmetric and positive definite matrix and $J > 0$, we have 
\begin{equation*}
S({\bf u}_t,{\bf u}) = S({\bf u},{\bf u}_t) \ \ \  \text{and} \ \ \ S({\bf u},{\bf u}) \geq 0.
\end{equation*}
The term $B$ that contains the boundary integrals is
\begin{multline}\label{bdry_integral_ref}
B({\bf u}_t,{\bf u}) = \int_{r^{(1)} = 0}^{r^{(1)} = 1} \int_{r^{(2)} = 0}^{r^{(2)} = 1}  \left[{\bf u}_t^T\bar{A}_3\bar{\nabla} {\bf u}\right]_{r^{(3)}=0}^{r^{(3)}=1} \ dr^{(1)}dr^{(2)}\\
+ \int_{r^{(2)} = 0}^{r^{(2)} = 1} \int_{r^{(3)} = 0}^{r^{(3)} = 1}  \left[{\bf u}_t^T\bar{A}_1\bar{\nabla} {\bf u}\right]_{r^{(1)}=0}^{r^{(1)}=1}\ dr^{(2)}dr^{(3)} +\int_{r^{(1)} = 0}^{r^{(1)} = 1} \int_{r^{(3)} = 0}^{r^{(3)} = 1}  \left[{\bf u}_t^T\bar{A}_2\bar{\nabla} {\bf u}\right]_{r^{(2)}=0}^{r^{(2)}=1} \ dr^{(1)}dr^{(3)},
\end{multline}
Comparing the terms (\ref{bdry3_curvi}) to (\ref{bdry2_curvi}) with (\ref{bdry_integral_ref}), we note that the intergral terms in (\ref{bdry_integral_ref}) are the scaled boundary traction,
\begin{multline}\label{bdry_integral_curvi}
B({\bf u}_t,{\bf u}) = \int_{r^{(1)} = 0}^{r^{(1)} = 1} \int_{r^{(2)} = 0}^{r^{(2)} = 1} J|\nabla r^{(3)}|\frac{1}{J\left|\nabla r^{(3)}\right|} \left[{\bf u}_t^T\bar{A}_3\bar{\nabla} {\bf u}\right]_{r^{(3)}=0}^{r^{(3)}=1} \ dr^{(1)}dr^{(2)}\\
+ \int_{r^{(2)} = 0}^{r^{(2)} = 1} \int_{r^{(3)} = 0}^{r^{(3)} = 1} J|\nabla r^{(1)}|\frac{1}{J\left|\nabla r^{(1)}\right|}  \left[{\bf u}_t^T\bar{A}_1\bar{\nabla} {\bf u}\right]_{r^{(1)}=0}^{r^{(1)}=1}\ dr^{(2)}dr^{(3)} \\
+\int_{r^{(1)} = 0}^{r^{(1)} = 1} \int_{r^{(3)} = 0}^{r^{(3)} = 1} J|\nabla r^{(2)}|\frac{1}{J\left|\nabla r^{(2)}\right|}  \left[{\bf u}_t^T\bar{A}_2\bar{\nabla} {\bf u}\right]_{r^{(2)}=0}^{r^{(2)}=1} \ dr^{(1)}dr^{(3)} = \int_{\partial \Omega} {\bf u}_t^T({\bf n}\cdot\mathcal{T})\ dS.
\end{multline}
Here, ${\bf n}$ is the unit outward norm of the cuvilinear domain $\Omega$. It is obvious that $B({\bf u}_t, {\bf u}) = 0$ if ${\bf u}$ satifies the homogeneous Dirichlt boundary condition ${\bf u} = {\bf 0}$ or the free surface boundary condition ${\bf n}\cdot\mathcal{T} = {\bf 0}$ in the direction $3$ provided periodic boundary conditions in the directions $1$ and $2$. The boundary term also vanishes at a grid interface if we impose continuity of ${\bf u}$ and continuity of the traction forcing ${\bf n}\cdot\mathcal{T}$. We then have 
\begin{equation}\label{simple_time_deri_curvi}
\frac{1}{2}\frac{d}{dt} \left(||\sqrt{\rho}{\bf u}_t||_2^2 + S({\bf u},{\bf u})\right) =  ({\bf u}_t,{\bf F})_2.
\end{equation}
Denote the continuous energy ${E} = ||\sqrt{\rho}{\bf u}_t||_2^2 + S({\bf u},{\bf u})$. From Cauchy-Schwarz inequality, we have
\begin{equation*}
({\bf u}_t,{\bf F})_2\leq ||{\bf u}_t||_2 ||{\bf F}||_2\leq \tilde {C}\sqrt{E}||{\bf F}||_2,
\end{equation*}
where $\tilde{C}$ is a constant. Then combine with (\ref{simple_time_deri_curvi}), we obtain
\begin{equation*}
\frac{d\sqrt{E}}{dt} \leq \tilde{C}||{\bf F}||_2.
\end{equation*}
Therefore, the total energy of the solution is non-inscreasing when the external force ${\bf F} = 0$ for all time.