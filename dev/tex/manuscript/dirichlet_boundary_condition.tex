%!TEX root = elastic_3d_sbp.tex
\subsubsection{Dirichlet boundary condition}
Next, we consider the Dirichlet boundary condition, 
\begin{equation}\label{dirichlet_boundary_condition}
{\bf u}(x^{(1)},x^{(2)},x^{(3)},t) = {\bf f}(x^{(1)},x^{(2)},x^{(3)},t),  \ \ \ t \geq 0,
\end{equation}
where $(x^{(1)},x^{(2)},x^{(3)})$ are on the top surface. It is obvious that we can set ${\bf u}_{{\bf i}_{\Omega_t}} = {\bf f}_{i_1,i_2}$, but this relation does not contain any ghost points. In order to involve the ghost points, instesd (\ref{dirichlet_boundary_condition}), we consider
\begin{equation}
{\bf u}_{tt}(x^{(1)},x^{(2)},x^{(3)},t) = {\bf f}_{tt}(x^{(1)},x^{(2)},x^{(3)},t),  \ \ \ t \geq 0.
\end{equation}
From the discrete point of view, we have
\begin{equation}\label{dbc_discrete}
({\bf u}_{tt})_{{\bf i}_{\Omega_t}}=\frac{1}{\rho_{{\bf i}_{\Omega_t}}} \hat{L}_h{\bf u}_{{\bf i}_{\Omega_t}} = ({\bf f}_{tt})_{i_1,i_2}.
\end{equation}
Since we have periodic boundary conditions in the directions $1,2$ and $D_3$ in the (\ref{discrete_elastic}) use the first derivative oprator which does not contain the ghost points, the only term contains the ghost points is 
\[\frac{\hat{G}_3(N_{33}){\bf u}_{{\bf i}_{\Omega_t}}}{\rho_{{\bf i}_{\Omega_t}}J_{{\bf i}_{\Omega_t}}}.\]
Then by using (\ref{dbc_discrete}), we have the values for ghost points ${\bf u}_{i,j,n_3+1}$. Since the initial condition is compatible with the boundary condition, we can integrate (\ref{dbc_discrete}) in time once to get
\[({\bf u}_t)_{{\bf i}_{\Omega_t}} = ({\bf f}_t)_{i_1,i_2}.\]
Therefore, we have a conservative energy $\frac{dE_h}{dt} = 0$ when ${\bf f}_t = {\bf 0}$.