%!TEX root = SISC_elastic_3d.tex
\section{The anisotropic elastic wave equation }
We consider the time dependent anisotropic elastic wave equation in three dimensional domain ${\bf x}\in\Omega$, where ${\bf x} = (x^{(1)},x^{(2)},x^{(3)})^T$. 
The domain $\Omega$ is partitioned into two subdomains $\Omega^f$ and $\Omega^c$ with an interface $\Gamma = \Omega^f\cap\Omega^c$. The material property is smooth in each subdomain, but maybe discontinuous at the interface $\Gamma$. Without loss of generality, we assume that the wave speed is slower in $\Omega^f$ than in $\Omega^c$, which motivates us to use a fine mesh in $\Omega^f$ and a coarse mesh in $\Omega^c$.  We further assume that both $\Omega^f$ and $\Omega^c$ have six, possibly curved, faces. Denote ${\bf r} = (r^{(1)},r^{(2)},r^{(3)})^T$ and  introduce smooth one-to-one mappings 
\[{\bf x}^{f} = {\bf X}^{f}({\bf r}) :  [0,1]^3 \rightarrow \Omega^{f} \subset \mathbb{R}^3 \]
and 
\[{\bf x}^{c} = {\bf X}^{c}({\bf r}) :  [0,1]^3 \rightarrow \Omega^{c} \subset \mathbb{R}^3\]
with ${\bf X}^{c}({\bf r}) = (x^{c,(1)}({\bf r}),x^{c,(2)}({\bf r}),x^{c,(3)}({\bf r}))^T$, ${\bf X}^{f}({\bf r}) = (x^{f,(1)}({\bf r}),x^{f,(2)}({\bf r}),x^{f,(3)}({\bf r}))^T$ to the subdomains $\Omega^f$ and $\Omega^c$,  respectively. Here, we want to mention that the interface $\Gamma$ corresponds to $r^{(3)} = 1$ for the coarse domain and $r^{(3)} = 0$ for the fine domain, that is we assume the fine domain $\Omega^f$ is on the top of coarse domain $\Omega^c$ in this paper. Then the elastic wave equation in the course domain $\Omega^c$ in terms of displacements can be written in curvilinear coordinates as
\begin{align}\label{elastic_curvi}
	%\rho^f\frac{\partial^2{\bf F}}{\partial^2 t} &= \frac{1}{J^f}\left[\partial_1(A_1^f\nabla{\bf F}) + \partial_2(A_2^f\nabla{\bf F}) +\partial_3(A_3^f\nabla{\bf F}) \right], \ \ \  {\bf r}\in\Omega^f_r,\ \ \  t\geq0,\nonumber\\
	\rho^c\frac{\partial^2{\bf C}}{\partial^2 t} = \frac{1}{J^c}\left[\partial_1(A_1^c\nabla{\bf C}) + \partial_2(A_2^c\nabla{\bf C}) +\partial_3(A_3^c\nabla{\bf C}) \right],\ \ \  {\bf r}\in[0,1]^3,\ \ \  t\geq0,
\end{align}
where ${\bf C}({\bf r}) = (C_1({\bf r}),C_2({\bf r}),C_3({\bf r}))^T$ is the three dimensional displacement vector and $\rho^c$ is the density function on coarse domain $\Omega^c$ repectively. In addition, we define
\begin{align*} 
	A_1^c\nabla{\bf C} &:= N_{11}^c\partial_1{\bf C} + N_{12}^c\partial_2{\bf C} + N_{13}^c\partial_3{\bf C}, \\
	A_2^c\nabla{\bf C} &:= N_{21}^c\partial_1{\bf C} + N_{22}^c\partial_2{\bf C} + N_{23}^c\partial_3{\bf C}, \\
	A_3^c\nabla{\bf C} &:= N_{31}^c\partial_1{\bf C} + N_{32}^c\partial_2{\bf C} + N_{33}^c\partial_3{\bf C},
\end{align*}
with $\partial_i = \frac{\partial}{\partial r_i}$, $\nabla = (\partial_1,\partial_2,\partial_3)$ and
\begin{equation}\label{N_definition}
	N_{ij}^c = J^c\big(O_i\big)^TM_{ij}^cO_j, \ \ i,j = 1,2,3,
\end{equation}
where, 
\[ O_{1}^T = \left(\begin{array}{cccccc}
1 & 0 & 0 &0 & 0 & 0\\
0 & 0 & 0 &0 & 0 & 1\\
0 & 0 & 0 &0 & 1 & 0\end{array}\right),  O_{2}^T = \left(\begin{array}{cccccc}
0 & 0 & 0 &0 & 0 & 1\\
0 & 1 & 0 &0 & 0 & 0\\
0 & 0 & 0 &1 & 0 & 0\end{array}\right),  O_{3}^T = \left(\begin{array}{cccccc}
0 & 0 & 0 &0 & 1 & 0\\
0 & 0 & 0 &1 & 0 & 0\\
0 & 0 & 1 &0 & 0 & 0\end{array}\right),\]
and $M_{ij}^c$ are $6\times 6$ stiffness matrices and they are symmetric and positive definite. In particular, for the isotropic elastic wave equation, we have
\[ M_{11}^c = \left(\begin{array}{ccc}
2\mu^c+\lambda^c & 0 & 0\\
0 & \mu^c & 0\\
0 & 0 & \mu^c\end{array}\right),\ \ \  M_{12}^c = \left(\begin{array}{ccc}
0 & \lambda^c & 0\\
\mu^c & 0 & 0\\
0 & 0 & 0\end{array}\right), \]
\begin{equation}\label{M_definition}
M_{22}^c = \left(\begin{array}{ccc}
\mu^c & 0 & 0\\
0 & 2\mu^c+\lambda^c & 0\\
0 & 0 & \mu^c\end{array}\right),\ \ \ M_{13}^c = \left(\begin{array}{ccc}
0 & 0 & \lambda^c\\
0 & 0 & 0\\
\mu^c & 0 & 0\end{array}\right),
\end{equation}
\[\ M_{33}^c = \left(\begin{array}{ccc}
\mu^c & 0 & 0\\
0 & \mu^c & 0\\
0 & 0 & 2\mu^c+\lambda^c\end{array}\right),\ \ \ M_{23}^c = \left(\begin{array}{ccc}
0 & 0 & 0\\
0 & 0 & \lambda^c\\
0 & \mu^c & 0\end{array}\right),\]
\[ M_{31}^c = (M_{13}^c)^T, \ \ \  M_{32}^c =(M_{23}^c)^T, \ \ \ M_{21}^c =(M_{12}^c)^T.\]
Here, $\lambda^c$ and $\mu^c$ are the first and second L{\' {a}}me parameters, respectively, and are determined by the material property. From the definitions of matrices $N_{ij}^c$ in (\ref{N_definition}), we have that $N_{ii}^c$, $i = 1,2,3$, are symmetric positive definite, and $N_{ij}^c=\big(N_{ji}^c\big)^T$, $i,j=1,2,3$. Finally, $J^c =$ det$\left({\bf a}^c_1,{\bf a}^c_2,{\bf a}^c_3\right)\in (0,\infty)$ is the Jacobian of transformation, with the derivative of the forward mapping, 
\begin{equation}\label{forward_map}
{\bf a}_k^c := \partial_k{\bf x}^c  = \left(\frac{\partial x^{c,(1)}}{\partial r^{k}},\frac{\partial x^{c,(2)}}{\partial r^{k}},\frac{\partial x^{c,(3)}}{\partial r^{k}}\right)^T,\ \ \ \ k = 1,2,3,
\end{equation}
and the backward mapping,
\begin{equation}\label{backward_map}
{\bf a}^{c,k} := \nabla_x^c r^{k} = \left(\frac{\partial r^{k}}{\partial x^{c,(1)}},\frac{\partial r^{k}}{\partial x^{c,(2)}},\frac{\partial r^{k}}{\partial x^{c,(3)}}\right)^T := \left(\xi_{1k}^c,\xi_{2k}^c,\xi_{3k}^c\right)^T,\ \ \ k = 1,2,3.
\end{equation}
 The metric relation between (\ref{forward_map}) and (\ref{backward_map}) is given by \cite{thompson1985numerical},
\begin{equation*}
{\bf a}^{c,i} = \frac{1}{J^c}\left({\bf a}_j^c\times{\bf a}_k^c\right), \ \ \ (i,j,k) \ \mbox{cycle}.
\end{equation*}
Denote the unit outward normal ${\bf n}_i^{c,\pm} = (n_i^{c,\pm,1},n_i^{c,\pm,2},n_i^{c,\pm,3})$, $i = 1,2,3$, for the boundaries of $r^{(i)}$ direction on subdomain $\Omega^c$, then
\begin{align}\label{outward_normal}
{\bf n}_i^{c,\pm} &:= (n^{c,\pm,1}_i,n^{c,\pm,2}_i,n^{c,\pm,3}_i)^T = \pm \frac{\nabla_x^c r^{(i)}}{\left|\nabla_x^c r^{(i)}\right|} \nonumber\\
&= \pm\frac{ \left(\xi_{1i}^c,\xi_{2i}^c,\xi_{3i}^c\right)^T}{\sqrt{(\xi_{1i}^c)^2+(\xi_{2i}^c)^2+(\xi_{3i}^c)^2}}.
\end{align}
Here, $'+'$ corresponds to $r^{(i)} = 1$ and $'-'$ corresponds to $r^{(i)} = 0$. The elastic wave equation in the fine domain $\Omega^f$ in terms of displacements in curvilinear coordinates is defined in the same way as in the  coarse domain $\Omega^c$. We have 
\begin{align}\label{elastic_curvi_f}
	\rho^f\frac{\partial^2{\bf F}}{\partial^2 t} &= \frac{1}{J^f}\left[\partial_1(A_1^f\nabla{\bf F}) + \partial_2(A_2^f\nabla{\bf F}) +\partial_3(A_3^f\nabla{\bf F}) \right], \ \ \  {\bf r}\in\Omega^f_r,\ \ \  t\geq0.
	%\rho^c\frac{\partial^2{\bf C}}{\partial^2 t} = \frac{1}{J^c}\left[\partial_1(A_1^c\nabla{\bf C}) + \partial_2(A_2^c\nabla{\bf C}) +\partial_3(A_3^c\nabla{\bf C}) \right],\ \ \  {\bf r}\in[0,1]^3,\ \ \  t\geq0,
\end{align}


At the interface $\Gamma$, suitable physical interface conditions are the continuity of the traction vectors and the continuity of the displacement vectors 
\begin{equation}\label{interface_cond}
\frac{A_3^f\nabla{\bf F}}{J^f|\nabla_x^f r^{(3)}|}  = \frac{A_3^c\nabla{\bf C}}{J^c\left|\nabla_x^c r^{(3)}\right|}, \ \ \ \ {\bf F} = {\bf C}.
\end{equation}
Together with suitable physical boundary conditions, the problem (\ref{elastic_curvi}, \ref{interface_cond}) is well-posed \cite{duru2014stable, petersson2015wave}.


