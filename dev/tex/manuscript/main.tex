\documentclass[a4paper]{article}

\usepackage[english]{babel}
\usepackage[utf8]{inputenc}
\usepackage{amsmath}
\usepackage{graphicx}
\usepackage[colorinlistoftodos]{todonotes}
\usepackage{tikz}
\usetikzlibrary{arrows}
\usepackage{booktabs}
\usepackage{threeparttable}
\usepackage{tikz}
\usetikzlibrary{arrows.meta}
\usepackage{pgfplots}
\usepackage{subcaption}
\usepackage[toc,page]{appendix}
\usepackage{bm}
\usepackage{amsfonts}
\usepackage{algorithm}
\usepackage[noend]{algpseudocode}

\newcommand{\wt}{\widetilde}


\title{Fourth order summation-by-parts finite difference methods for  3-D elastic wave propagation in curvilinear coordinates with mesh refinement interfaces}

\date{\today}
\author{ Lu Zhang \and Siyang Wang \and N. Anders Petersson}
\begin{document}
\maketitle

\begin{abstract}
We analyze
\end{abstract}

\section{Introduction}

\section{The isotropic elastic wave equation }
We consider the time dependent isotropic elastic wave equation in three dimensional domain for simple Cartesian domain and the curvilinear domian.
\subsection{The Cartesian coordinates}
The problem is defined on the domain ${\bf x}\in\Omega = [0,a^{(1)}]\times[0,a^{(2)}]\times[0,a^{(3)}]$ with ${\bf x} = (x_1,x_2,x_3)^T$ are Cartesian coordinates. Denote ${\bf u} = (u_1,u_2,u_3)^T$ to be the three dimensional displacement vector in Cartesian coordinates, then the elastic wave equation takes the form,
\begin{align*}
    \rho\frac{\partial^2{\bf u}}{\partial^2 t} &= \nabla\cdot\mathcal{T} + {\bf F}, \ \ \ {\bf x}\in\Omega,\ \ \ t\geq 0,\\
    \nabla\cdot\mathcal{T} &:= {\bf Lu},
\end{align*}
provided with appropriate initial and boundary conditions. Here, $\rho$ is density, $\mathcal{T}$ is the stress tensor and ${\bf F}$ is the force function. The spatial operator ${\bf L}$ is called $3\times3$ symmetric Kelvin-Christoffel differential operator matrix, specifically,
\begin{equation*}
    {\bf L u} = \partial_1(A_1\nabla{\bf u}) + \partial_2(A_2\nabla{\bf u}) + \partial_3(A_3\nabla{\bf u}),
\end{equation*}
with
\begin{align*}
A_1\nabla{\bf u} &:= M_{11}\partial_1{\bf u} + M_{12}\partial_2{\bf u} + M_{13}\partial_3{\bf u}, \\
A_2\nabla{\bf u} &:= M_{21}\partial_1{\bf u} + M_{22}\partial_2{\bf u} + M_{23}\partial_3{\bf u}, \\
A_3\nabla{\bf u} &:= M_{31}\partial_1{\bf u} + M_{32}\partial_2{\bf u} + M_{33}\partial_3{\bf u},
\end{align*}
where $M_{ij}, i,j = 1,2,3$ are defined by
\begin{equation*}
M_{ij} = P^T_iCP_j.
\end{equation*}
Here, $C$ is symmetric and positive definite, we refer to Appendix ? for the definitions of matrices $C$ and $P_i, i = 1,2,3$. Especially, for the isotropic elastic wave equation, we have
\[ M_{11} = \left(\begin{array}{ccc}
2\mu+\lambda & 0 & 0\\
0 & \mu & 0\\
0 & 0 & \mu\end{array}\right), M_{12} = \left(\begin{array}{ccc}
0 & \lambda & 0\\
\mu & 0 & 0\\
0 & 0 & 0\end{array}\right), M_{13} = \left(\begin{array}{ccc}
0 & 0 & \lambda\\
0 & 0 & 0\\
\mu & 0 & 0\end{array}\right),\]
\[ M_{21} =(M_{12})^T, M_{22} = \left(\begin{array}{ccc}
\mu & 0 & 0\\
0 & 2\mu+\lambda & 0\\
0 & 0 & \mu\end{array}\right), M_{23} = \left(\begin{array}{ccc}
0 & 0 & 0\\
0 & 0 & \lambda\\
0 & \mu & 0\end{array}\right),\]
\[ M_{31} = (M_{13})^T, \ \ \ \ \ M_{32} =(M_{23})^T, \ \ \ \ \ M_{33} = \left(\begin{array}{ccc}
\mu & 0 & \lambda\\
0 & \mu & 0\\
0 & 0 & 2\mu+\lambda\end{array}\right),\]
Here, $\lambda$ and $\mu$ are the first and second Lame parameters respectively, which are determined by the properties of the materials.

Denote the outward unit normal ${\bf n}_i = (n_i^{(1)},n_i^{(2)},n_i^{(3)})$ for the boundaries $x^{(i)} = 0, a^{(i)}, i = 1,2,3$. For example, ${\bf n}_1 = (\pm 1, 0,0)$ for the boundaries $x^{(1)} = 1$ or $x^{(1)} = 0$, then we have the boundary traction data 
\begin{equation*}
{\bf n}_i\cdot\mathcal{T} = n^{(1)}_iA_1\nabla{\bf u} + n^{(2)}_iA_2\nabla{\bf u} + n^{(3)}_iA_3\nabla{\bf u},\ \ i = 1,2,3. 
\end{equation*}
A homogeneous Dirichlet boundary condition corresponds to ${\bf u} = {\bf 0}$ and a free surface boundary condtion is ${\bf n}_i\cdot\mathcal{T}  = {\bf 0}, i = 1,2,3$.

It is known that the elastic wave equation with the homogeneous Dirichelet boundary conidtion or free surface boundary condition with a Cartesian domain is a well-posed problem and the total energy of the solution is conserved if the external force function ${\bf F} = 0$, we refer to \cite{?} for a detailed analysis.

\subsection{Generalization to curvilinear coordinates}
%Present the transformed equation. Notation is important. We can write a few formulas first, and discuss if it is good notation. Maybe we shall follow notations from one of Anders' papers?

In this section, we consider a curvilinear domain. Assume there is a one-to-one mapping ${\bf x} = {\bf x}({\bf r}) : [0,1]^3 \rightarrow \Omega \in \mathbb{R}^3$ with ${\bf x}({\bf r}) = (x^{(1)}({\bf r}),x^{(2)}({\bf r}),x^{(3)}({\bf r}))^T, {\bf r} = (r^{(1)},r^{(2)},r^{(3)})^T, 0\leq r^{(i)}\leq1, i = 1,2,3$.

We define the derivative of the forward mapping to be 
\begin{equation}\label{elastic_eq_carte}
{\bf a}_k := \bar{\partial}_k{\bf x} = \left(\frac{\partial x^{(1)}}{\partial r^{(k)}},\frac{\partial x^{(2)}}{\partial r^{(k)}},\frac{\partial x^{(3)}}{\partial r^{(k)}}\right)^T,
\end{equation}
for $k = 1,2,3$ and the backward mapping to be
\begin{equation*}
{\bf a}^k := \nabla r^{(k)} = \left(\frac{\partial r^{(k)}}{\partial x^{(1)}},\frac{\partial r^{(k)}}{\partial x^{(2)}},\frac{\partial r^{(k)}}{\partial x^{(3)}}\right)^T := (\xi_{1k},\xi_{2k},\xi_{3k})^T,
\end{equation*}
for $k = 1,2,3$. It is known that the backward mapping can be expressed by the forward mapping \cite{?}, 
\begin{equation*}
{\bf a}^i = \frac{1}{J}({\bf a}_j\times{\bf a}_k), \ \ \ (i,j,k) \ \text{cycle},
\end{equation*}
here, $J$ is the Jacobian of the forward mapping $J = \text{det}({\bf a}_1, {\bf a}_2, {\bf a}_3)$ and the mapping is assumed to be non-singular with $0<J<\infty$.

After applying the forward and backward mapping, reader can refer to \cite{?} for details, the elastic wave equation (\ref{elastic_eq_carte}) can be written as
\begin{equation}\label{elastic_eq_curvi}
\rho\frac{\partial^2 {\bf u}}{\partial t^2} = \frac{1}{J}\left[\tilde{\partial }_1(\tilde{A}_1\tilde{\nabla}{\bf u}) + \tilde{\partial }_2(\tilde{A}_2\tilde{\nabla}{\bf u}) +\tilde{\partial }_3(\tilde{A}_3\tilde{\nabla}{\bf u}) \right] + {\bf F},
\end{equation}
where
\begin{align*}
	\bar{A}_1\bar{\nabla}{\bf u} &:= N_{11}\bar{\partial}_1{\bf u} + N_{12}\bar{\partial}_2{\bf u} + N_{13}\bar{\partial}_3{\bf u}, \\
	\bar{A}_2\bar{\nabla}{\bf u} &:= N_{21}\bar{\partial}_1{\bf u} + N_{22}\bar{\partial}_2{\bf u} + N_{23}\bar{\partial}_3{\bf u}, \\
	\bar{A}_3\bar{\nabla}{\bf u} &:= N_{31}\bar{\partial}_1{\bf u} + N_{32}\bar{\partial}_2{\bf u} + N_{33}\bar{\partial}_3{\bf u},
\end{align*}
here, 
\begin{equation}\label{definition_Nij}
N_{ij} = J\bar{P}_i^TC\bar{P}_j \ \ \text{with} \ \ \bar{P}_i = \sum_{j=1}^3\xi_{ji}P_j.
\end{equation}
The matrices $C$ and $P_i, i = 1,2,3$ can be found in Appendix ?. 

The transformation of homogeneous Dirichlet boundary condiiton to curvilinear coordinates is straightfroward, ${\bf u}({\bf x}) = {\bf u}(\bf r)$. To transfer the free surface boundary condition to curvilinear coordinates, we firstly write the unit outside normal to be the function of metric derivatives, for example, along the boundary $r^{(3)} = 1$ or $r^{(3)} = 0$,
\begin{equation}\label{definition_bdry3_curvi}
{\bf n}_3 := (n^{(1)}_3,n^{(2)}_3,n^{(3)}_3)^T = \pm \frac{\nabla r^{(3)}}{\left|\nabla r^{(3)}\right|} = \frac{\pm 1}{\sqrt{((\xi_{13})^2+(\xi_{23})^2+(\xi_{33})^2)}}\left(\xi_{13},\xi_{23},\xi_{33}\right)^T,
\end{equation}
here, $'+'$ corresponds to $r^{(3)} = 1$ and $'-'$ corresponds to $r^{(3)} = 0$. Then after some straightforward calculation, the traction free boundary condition can be written as
\begin{equation}\label{bdry3_curvi}
\mathcal{T}\cdot{\bf n}_3 = \frac{\pm 1}{J\left|\nabla r^{(3)}\right|}\bar{A}_3\bar{\nabla}{\bf u}, \ \ \ r^{(3)} = 0, 1.
\end{equation}
Similarly, we can derive the traction free boundary condition along $r^{(1)} = 0,1$ and $r^{(2)} = 0,1$ as
\begin{equation}\label{bdry1_curvi}
\mathcal{T}\cdot{\bf n}_1 = \frac{\pm 1}{J\left|\nabla r^{(1)}\right|}\bar{A}_1\bar{\nabla}{\bf u}, \ \ \ r^{(1)} = 0, 1,
\end{equation}
\begin{equation}\label{bdry2_curvi}
\mathcal{T}\cdot{\bf n}_2 = \frac{\pm 1}{J\left|\nabla r^{(2)}\right|}\bar{A}_2\bar{\nabla}{\bf u}, \ \ \ r^{(2)} = 0, 1,
\end{equation} 
respectively. Here, the ${\bf n}_1$ and ${\bf n}_2$ have a similar definition as ${\bf n}_3$ in (\ref{definition_bdry3_curvi}).

\subsubsection{Energy estimate}

From the definition of matrcies $N_{ij}$, we can easily verify that $N_{11}, N_{22}, N_{33}$ are positive definite and $N_{ij} = N_{jk}^T$. In the curvilinear coordinates, we define the $L^2$ scalar product of the two real vector-valued functions ${\bf u}({\bf r})\in\mathbb{R}^3\rightarrow \mathbb{R}^3$ and ${\bf v}(\bf r)\in\mathbb{R}^3\rightarrow \mathbb{R}^3$ by
\begin{equation*}
({\bf u},{\bf v})_2 = \int_{{\bf r}\in[0,1]^3} \left(\sum_{l=1}^q u^{l}v^{l}\right)Jdr^{(1)}dr^{(2)}dr^{(3)}.
\end{equation*}
To analyze the energy estimate of the solution of an elastic wave equation in the curvilinear domain, we multiply the equation (\ref{elastic_eq_curvi}) by $J{\bf u}_t$ and integrate over the parameter space $[0,1]^3$,
\begin{align}\label{time_deri_curvi}
({\bf u}_t,\rho {\bf u}_{tt})_2 &= \left({\bf u}_t,\frac{1}{J}\bar{\partial}_1(\bar{A}_1\bar{\nabla}{\bf u})\right)_2+ \left({\bf u}_t,\frac{1}{J}\bar{\partial}_2(\bar{A}_2\bar{\nabla}{\bf u})\right)_2+ \left({\bf u}_t,\frac{1}{J}\bar{\partial}_3(\bar{A}_3\bar{\nabla}{\bf u})\right)_2+({\bf u}_t, {\bf F})_2\nonumber\\
&:= -S({\bf u}_t,{\bf u}) + B({\bf u}_t,{\bf u}) + ({\bf u}_t,{\bf F})_2,
\end{align}
where the term $S$ represents the interior terms after integration by parts,
\begin{multline*}
S({\bf u}_t,{\bf u}) = \int_{{\bf r}\in[0,1]^3} (\bar{\partial}_1 {\bf u}_t)^T ({\bar A}_1\bar{\nabla}{\bf u}) + (\bar{\partial}_2 {\bf u}_t)^T ({\bar A}_2\bar{\nabla}{\bf u}) + (\bar{\partial}_3 {\bf u}_t)^T ({\bar A}_3\bar{\nabla}{\bf u})\ dr^{(1)}dr^{(2)}dr^{(3)}\\
 = \int_{{\bf r}\in[0,1]^3} (\bar{\partial}_1 {\bf u}_t)^T (N_{11}\bar{\partial}_1{\bf u} +N_{12}\bar{\partial}_2{\bf u}+ N_{13}\bar{\partial}_3{\bf u}) + (\bar{\partial}_2 {\bf u}_t)^T (N_{21}\bar{\partial}_1{\bf u} +N_{22}\bar{\partial}_2{\bf u}+ N_{23}\bar{\partial}_3{\bf u}) \\
 +(\bar{\partial}_3 {\bf u}_t)^T (N_{31}\bar{\partial}_1{\bf u} +N_{32}\bar{\partial}_2{\bf u}+ N_{33}\bar{\partial}_3{\bf u})\ dr^{(1)}dr^{(2)}dr^{(3)}\\
 = \sum_{i = 1}^3\sum_{j=1}^3  \int_{{\bf r}\in[0,1]^3} (\bar{\partial}_i {\bf u}_t)^T(N_{ij}\bar{\partial}_j {\bf u})\ dr^{(1)}dr^{(2)}dr^{(3)}.
\end{multline*}
Recall the definition of $N_{ij}, i,j = 1,2,3$ in (\ref{definition_Nij}), we have
\begin{multline*}
S({\bf u}_t,{\bf u}) = \sum_{i=1}^3\sum_{j=1}^3 \int_{{\bf r}\in[0,1]^3} (\bar{\partial}_i {\bf u}_t)^T(J\bar{P}_i^TC\bar{P}_j\bar{\partial}_j {\bf u})\ dr^{(1)}dr^{(2)}dr^{(3)} \\
= \sum_{i=1}^3\sum_{j=1}^3 \int_{{\bf r}\in[0,1]^3} (\bar{P}_i\bar{\partial}_i {\bf u}_t)^TC(\bar{P}_j\bar{\partial}_j {\bf u}) J\ dr^{(1)}dr^{(2)}dr^{(3)}.
\end{multline*}
Since $C$ is a symmetric and positive definite matrix and $J > 0$, we have 
\begin{equation*}
S({\bf u}_t,{\bf u}) = S({\bf u},{\bf u}_t) \ \ \  \text{and} \ \ \ S({\bf u},{\bf u}) \geq 0.
\end{equation*}
The term $B$ that contains the boundary intergrals satisfies
\begin{multline}\label{bdry_integral_ref}
B({\bf u}_t,{\bf u}) = \int_{r^{(1)} = 0}^{r^{(1)} = 1} \int_{r^{(2)} = 0}^{r^{(2)} = 1}  \left[{\bf u}_t^T\bar{A}_3\bar{\nabla} {\bf u}\right]_{r^{(3)}=0}^{r^{(3)}=1} \ dr^{(1)}dr^{(2)}\\
+ \int_{r^{(2)} = 0}^{r^{(2)} = 1} \int_{r^{(3)} = 0}^{r^{(3)} = 1}  \left[{\bf u}_t^T\bar{A}_1\bar{\nabla} {\bf u}\right]_{r^{(1)}=0}^{r^{(1)}=1}\ dr^{(2)}dr^{(3)} +\int_{r^{(1)} = 0}^{r^{(1)} = 1} \int_{r^{(3)} = 0}^{r^{(3)} = 1}  \left[{\bf u}_t^T\bar{A}_2\bar{\nabla} {\bf u}\right]_{r^{(2)}=0}^{r^{(2)}=1} \ dr^{(1)}dr^{(3)},
\end{multline}
Compare the terms (\ref{bdry3_curvi}) to (\ref{bdry2_curvi}) with (\ref{bdry_integral_ref}), we note that the intergral terms in (\ref{bdry_integral_ref}) are the scaled boundary traction,
\begin{multline}\label{bdry_integral_curvi}
B({\bf u}_t,{\bf u}) = \int_{r^{(1)} = 0}^{r^{(1)} = 1} \int_{r^{(2)} = 0}^{r^{(2)} = 1} J|\nabla r^{(3)}|\frac{1}{J\left|\nabla r^{(3)}\right|} \left[{\bf u}_t^T\bar{A}_3\bar{\nabla} {\bf u}\right]_{r^{(3)}=0}^{r^{(3)}=1} \ dr^{(1)}dr^{(2)}\\
+ \int_{r^{(2)} = 0}^{r^{(2)} = 1} \int_{r^{(3)} = 0}^{r^{(3)} = 1} J|\nabla r^{(1)}|\frac{1}{J\left|\nabla r^{(1)}\right|}  \left[{\bf u}_t^T\bar{A}_1\bar{\nabla} {\bf u}\right]_{r^{(1)}=0}^{r^{(1)}=1}\ dr^{(2)}dr^{(3)} \\
+\int_{r^{(1)} = 0}^{r^{(1)} = 1} \int_{r^{(3)} = 0}^{r^{(3)} = 1} J|\nabla r^{(2)}|\frac{1}{J\left|\nabla r^{(2)}\right|}  \left[{\bf u}_t^T\bar{A}_2\bar{\nabla} {\bf u}\right]_{r^{(2)}=0}^{r^{(2)}=1} \ dr^{(1)}dr^{(3)} = \int_{\partial \Omega} {\bf u}_t^T({\bf n}\cdot\mathcal{T})\ dS.
\end{multline}
Here, ${\bf n}$ is the unit outward norm of the cuvilinear domain $\Omega$. It is obviously that $B({\bf u}_t, {\bf u}) = 0$ if ${\bf u}$ satifies the homogeneous Dirichlt boundary condition ${\bf u} = {\bf 0}$ or the free surface boundary condition ${\bf n}\cdot\mathcal{T} = {\bf 0}$. Finally, we rewrite (\ref{time_deri_curvi}) to
\begin{equation*}
\frac{1}{2}\frac{d}{dt} \left(||\sqrt{\rho}{\bf u}_t||_2^2 + S({\bf u},{\bf u})\right) = B({\bf u}_t,{\bf u}) + ({\bf u}_t,{\bf F})_2,
\end{equation*}
here, the term $||\sqrt{\rho}{\bf u}_t||_2^2$ represents the kinematic energy and the term $S({\bf u},{\bf u})$ is so called potental energy. If we have a homegeneous Dirichlet bounary condition or a free suface boundary condition, then
\begin{equation}\label{simple_time_deri_curvi}
\frac{1}{2}\frac{d}{dt} \left(||\sqrt{\rho}{\bf u}_t||_2^2 + S({\bf u},{\bf u})\right) =  ({\bf u}_t,{\bf F})_2,
\end{equation}
Integrating (\ref{simple_time_deri_curvi}) in time gives
\begin{equation*}
\frac{1}{2}\left(||\sqrt{\rho}{\bf u}_t||_2^2 + S({\bf u},{\bf u})\right)  = \frac{1}{2}\left(||\sqrt{\rho}{\bf u}_t||_2^2 + S({\bf u},{\bf u})\right)\Big|_{t = 0} + \int_{t = 0}^{t = T} ({\bf u}_t,{\bf F})_2 \ dt.
\end{equation*}
Therefore, the total energy of the solution of the elastic wave equation is conserved if the external force ${\bf F} = {\bf 0}$ for all time.


\section{The spatial discretization}
%We present the SBP operators, the semi-discrete equation, and the discretized boundary conditions and interface conditions. Also a semi-discrete energy estimate (if we do not write a fully discrete energy estimate).

In this section, we only describe the discretization for the curvilinear domain, and the Catesian domain can be obtained by simply setting the Cartesian coordinates as $x^{(k)}(r^{(k)}) = a^{(k)}r^{(k)} $ with $a^{(k)}$ being constants.
\subsection{SBP operators in $1$D}
Consider a uniform discretization of the domain $x\in[0,1]$ with the grids,
\[\wt
{{\bf x}} = [x_0,x_1,\cdots,x_n,x_{n+1}]^T,\ \  x_i = (i-1)h,\ \ i = 0,1,\cdots,n,n+1,\ \ h = 1/(n-1),\]
where $i = 1,n$ correspond to the grid points on the boundary, and $i = 0,n+1$ are ghost points outside the physical domain.
We also denote ${\bf x} = [x_1,x_2,\cdots,x_n]^T$ the grid that does not  contain the ghost points. We adopt the same notation as in \cite{?} by using the tilde symbol to indicate that ghost points are included. The  operator $D \approx \frac{\partial }{\partial x}$ is a first derivative SBP operator if 
\begin{equation}\label{first_sbp}
({\bf u}, D{\bf v})_h = -(D{\bf u},{\bf v})_h - u_1v_1 + u_nv_n,
\end{equation}
with a scalar product
\begin{equation}\label{inner_product}
({\bf u},{\bf v})_h = h\sum_{i = 1}^{n}\omega_iu_iv_i.
\end{equation}
Here, $0<\omega_i < \infty $ are the weights of scalar product. The SBP operator $D$ has a centered difference stencil at the grid points away from the boudnary and the corresponding weights $\omega_i = 1$. To satify the SBP identity (\ref{first_sbp}), the coefficients in $D$ are  modified at a few points near the boundary and the corresponding weights $\omega_i \neq 1$.

To discretize the elastic wave equation, we also need to approximate the second derivative with variable coefficient $(\gamma(x)u_x)_x$. Here, the known function $\gamma(x)>0$ describes the property of the material. There are two different fourth order accurate SBP operators for the approximation of $(\gamma(x)u_x)_x$. The first one $\wt{G}(\gamma){\bf u} \approx (\gamma(x)u_x)_x $, derived by Sj\"ogreen and Petersson \cite{?}, uses one ghost point outside each boundary, and satisfies the second derivative SBP identity,
\begin{equation}\label{sbp_2nd_1}
({\bf u}, \wt{G}(\gamma){\bf v})_h = -S_\gamma({\bf u},{\bf v})-u_1\gamma_1\wt{\bf b}_1{\bf v} + u_n\gamma_n\wt{\bf b}_n {\bf v}.
\end{equation}
Here, the bilinear form $S_\gamma(\cdot,\cdot)$ is symmetric and positive semi-definite, and does not use any ghost points. The operators $\wt{\bf b}_1$ and $\wt{\bf b}_n$ aprroximate the first derivative on the left and right boundaries, respectively. Using the left boundary as an example, we have 
\[
\wt{\bf b}_1 {\bf v} = \frac{1}{h}\sum_{i=0}^{4} \wt{d}_i v_i,
\]
as the fourth order accurate approximation of $u_x(x_1)$. We note that the notation $\wt{G}(\gamma){\bf v}$ implies that the operator $\wt{G}$ uses ${\bf v}$ on all grid points $\wt{{\bf x}}$, but $\wt{G}(\gamma){\bf v}$ only returns values on the grid ${\bf x}$ without ghost points. Therefore, when writing in matrix form, $\wt{G}$ is a non-square matrix of size $n$ by $n+2$.

The other SBP operator ${G}(\gamma){\bf u} \approx (\gamma(x)u_x)_x $ is developed by Mattsson \cite{?} without using any ghost points, and satisfies a similar SBP identity,
\begin{equation}\label{sbp_2nd_2}
({\bf u}, G(\gamma){\bf v})_h = -S_\gamma({\bf u},{\bf v})-u_1\gamma_1{\bf b}_1{\bf v} + u_n\gamma_n{\bf b}_n{\bf v}.
\end{equation}
Here, ${\bf b}_1$ and ${\bf b}_n$ are also finite difference operators for the first derivative at the boundaries, but are constructed to third order accurate,
\[
{\bf b}_1 {\bf v} = \frac{1}{h}\sum_{i=1}^{4} d_i v_i. 
\]
In this case, ${G}(\gamma)$ is square in matrix form. 


For the second derivative SBP operators $\wt{G}(\gamma)$ and $G(\gamma)$, both of them use a fourth order five points centered difference stencil to approximate $(\gamma u_x)_x$ for the interior points away from the boundaries. For the first and the last six grid points close to the boudaries, the operators $G(\gamma)$ and $\wt{G}(\gamma)$ use second order accurate one-sides difference stencils. They are designed to satisfy (\ref{sbp_2nd_2}) and (\ref{sbp_2nd_1}), respectively. In the following section, we use both of them to develop a multi-block finite difference discretization for the elastic wave equation. 

%For stucture and the form of the operators $\tilde{G}(\gamma)\tilde {\bf u}$ and $G(\gamma){\bf u}$, we refer the readers \cite{???} for detailed information.

\subsection{Semi-discretization of the elastic wave equation}
We consider the elastic wave equation in curvilinear coordinates 
\[\rho\frac{\partial^2 {\bf u}}{\partial t^2} = L{\bf u},\]
where
\begin{multline}\label{spatial_operator}
L{\bf u} = \frac{1}{J}\Big[\bar{\partial}_1(N^{11}\bar{\partial}_1{\bf u}) + \bar{\partial}_2(N^{22}\bar{\partial}_2{\bf u}) + \bar{\partial}_3(N^{33}\bar{\partial}_3{\bf u}) + \bar{\partial}_1(N^{12}\bar{\partial}_2{\bf u}) + \bar{\partial}_1(N^{13}\bar{\partial}_3{\bf u}) \\
+ \bar{\partial}_2(N^{21}\bar{\partial}_1{\bf u}) + \bar{\partial}_2(N^{23}\bar{\partial}_3{\bf u}) +\bar{\partial}_3(N^{31}\bar{\partial}_1{\bf u}) + \bar{\partial}_3(N^{32}\bar{\partial}_2{\bf u})\Big].
\end{multline}
The parameter space $[0,1]\times[0,1]\times[0,1]$ is discretized as $r_i^{(1)} = (i-1)h_1, i = 0,1,2,\cdots,n_1+1$, $r_i^{(2)} = (i-1)h_2, i = 0,1,2,\cdots,n_2+1$, and $r_i^{(3)} = (i-1)h_3, i = 0,1,2,\cdots,n_3+1$ with $h_1 = 1/(n_1-1), h_2 = 1/(n_2-1), h_3 = 1/(n_3-1)$, the ghost points are used to impose the boundary conditions and interface conditions.

The terms $\bar{\partial}_i(N^{ii}\bar{\partial}_i{\bf u}), i = 1,2,3$ in the spatial operator (\ref{spatial_operator}) are approximated by a fourth order second derivative SBP4 operator. The operators $D_i$ are used to approximate the $\bar{\partial}_i, i = 1,2,3$. The terms $\bar{\partial}_i(N^{ij}\bar{\partial}_j{\bf u}), i = 1,2,3, j = 1,2,3$ are approximated by using the first derivative operatoe $D$ twice. Then the spatial operator (\ref{spatial_operator}) is discretizes as
\begin{multline}
L_h{\bf u}_{i,j,k} = \frac{1}{J_{i,j,k}}\big[G_1(N_{11}){\bf u}_{i,j,k}+G_2(N_{22}){\bf u}_{i,j,k}+\tilde{G}_3(N_{33}){\bf u}_{i,j,k}+D_1(N_{12}D_2{\bf u}_{i,j,k})\\
+D_1(N_{13}D_3{\bf u}_{i,j,k})+D_2(N_{21}D_1{\bf u}_{i,j,k})+D_2(N_{23}D_3{\bf u}_{i,j,k})+D_3(N_{31}D_1{\bf u}_{i,j,k})\\
+D_3(N_{32}D_2{\bf u}_{i,j,k})\big],
\end{multline}
for $i = 1,2,\cdots,n_1$, $j = 1,2,\cdots,n_2$ and $k = 1,2,\cdots,n_3$.
 
\subsection{Physical boundary conditions}
We consier both homegeneous Dirichlet boundary condition and free surface bounary condition.

\section{Grid refinement interface}
In this section, we partition the physical domian into two subdomains such that the discontinuity is aligned with subdomian boundary $\Gamma$,
\begin{equation}\label{coarse_problem}
\rho^c\frac{\partial ^2 {\bf u}^c}{\partial t^2} = {\bf Lu}^c, \ \ \ \ \ {\bf x} \in \Omega_c,\ \ \ t>0,
\end{equation}
\begin{equation}\label{fine_problem}
\rho^f\frac{\partial ^2 {\bf u}^f}{\partial t^2} = {\bf Lu}^f, \ \ \ \ \ {\bf x} \in \Omega_f,\ \ \ t>0
\end{equation}
provided suitable initial and boundary conditions. We assume that $\Omega_f$ is on the top of $\Omega_c$ with $\Omega_c\cup\Omega_f = \Omega$ and $\Omega_c\cap\Omega_f = \Gamma$. The interface conditions are given to guarantee the continuity of the solutions and the continuity of the traction force,
\begin{equation}\label{continuity_sol}
{\bf u}_f = {\bf u}_c, \ \ \ \ \ \ \ \ \ \ \ \ \ \ \ \ {\bf x}\in \Gamma, \ \ \ t>0, 
\end{equation}
\begin{equation}\label{continuity_trac}
\mathcal{T}^f\cdot{\bf n}^f = -\mathcal{T}^c\cdot{\bf n}^c,  \ \ \ \ \ \ \  {\bf x}\in \Gamma, \ \ \ t>0.
\end{equation}
For simplicity, we consider periodic boundary condition in directions $1$ and $2$, the Dirichlet boundary condition on the bottom surface (direction $3$) and traction force on the top surface (direction $3$).

\subsection{The fourth order SBP scheme}\label{sub_section_4_1}
We approximate the elastic wave equations (\ref{coarse_problem}) and (\ref{fine_problem}) by two different $4$th order SBP schemes. Specifically, we use a Cartesian mesh with mesh size $h_i^c (h_i^f)$ in the coarse domain $\Omega_c$ (fine domain $\Omega_f$) for direction $i$ with $n_i^c = 1/h_i^c +1 (n_i^f = 1/h_i^f +1), i = 1,2,3$.

Suppose $(r^{c,(1)}_i, r^{c,(2)}_j, r^{c,(3)}_k), i = 1,2,\cdots,n_1^c, j = 1,2,\cdots,n_2^c,k=0,1,\cdots,n_3^c+2$ are grid points in the coase doamin $\Omega_c$ and   $(r^{f,(1)}_i, r^{f,(2)}_j, r^{f,(3)}_k), i = 1,2,\cdots,n_1^f, j = 1,2,\cdots,n_2^f,k=1,\cdots,n_3^f+2$ are grid points in the fine doamin $\Omega_f$ after spacial discretization. For the coasre domain, we consider $2n^c_1n^c_2$ ghost points in the direction $3$ with $k = 0, n_3^c+2$ in $r^{c,(3)}_k$ for coarse domain $\Omega_c$ and $n^f_1n^f_2$ ghost points in direction $3$ with $k = 1,n_3^c+1$ in $r^{f,(3)}_k$ for fine domain $\Omega_f$.

For the interface conditions, we impose $n_1^cn_2^c$ linear equations that comes from the coarse domain $\Omega_c$. Precisely, we use the SBP operator (\ref{?}) which contains the ghost points to approximate the spatial derivative in direction $3$ for the coasre domain $\Omega_c$ (both interior and boudnaries); along the interface $\Gamma$, we use the SBP operator (\ref{?}) which does not contain any ghost points to approximate the spatial derivative in direction $3$ for the fine domain $\Omega_f$, while we use the SBP operator $(\ref{?})$ which contains the ghost points to approxiamte the spatial derivative in direction $3$ for other points in fine domain $\Omega_f$.

Firstly, we consider the term $\bar{\partial}_3(N_{33}\bar{\partial}_3{\bf u})$. For the grids which lying on the interface $\Gamma$, we approximate the therm $\bar{\partial}_3(N_{33}^{c}\bar{\partial}_3{\bf u}^{c})$ by applying the SBP operator $\tilde{G}^c_3(N_{33}^c)$ and the term $\bar{\partial}_3(N_{33}^{f}\bar{\partial}_3{\bf u}^{f})$ by applying the SBP operator $G^f_3(N_{33}^f)$. For the other grids, we approximate the therm $\bar{\partial}_3(N_{33}^{c,f}\bar{\partial}_3{\bf u}^{c,f})$ by applying the SBP operator $\tilde{G}^{c,f}_3(N_{33}^{c,f})$. For the other terms in (\ref{spatial_operator}), we approximate the terms $\bar{\partial}_i(N_{ii}^{c,f}\bar{\partial}_i{\bf u}^{c,f}), i = 1,2$, $\bar{\partial}_i(N_{ij}\bar{\partial}_j{\bf u}), i,j = 1,2$ by applying the central difference operator $D_i,i=1,2$ twice; approximate the term $\bar{\partial}_i(N_{ij}\bar{\partial}_j{\bf u}), i = 1,2,j = 3$ or $i = 3, j = 1,2$ by applying the SBP operator $D_3$ without ghost points in the direction $3$ and central difference operator $D_i$ in the direction $1,2$. Specifically, for (\ref{coarse_problem}), we have
\begin{multline}\label{coarse_scheme}
\rho_{i,j,k}^c ({\bf u}_{tt}^c)_{i,j,k} = \frac{1}{J_{i,j,k}^c}\Big[D_1^c(N_{11}^cD_1^c{\bf u}_{i,j,k}^c)+D_2^c(N_{22}^cD_2^c{\bf u}_{i,j,k}^c)+\tilde{G}_3^c(N_{33}^c){\bf u}_{i,j,k}^c\\
+D_1^c(N_{12}^cD_2^c{\bf u}_{i,j,k}^c)+D_1^c(N_{13}^cD_3^c{\bf u}_{i,j,k}^c)+D_2^c(N_{21}^cD_1^c{\bf u}_{i,j,k}^c)+D_2^c(N_{23}^cD_3^c{\bf u}_{i,j,k}^c)\\
+D_3^c(N_{31}^cD_1^c{\bf u}_{i,j,k}^c)+D_3^c(N_{32}^cD_2^c{\bf u}_{i,j,k}^c)\Big] \\
:= \tilde{M}_c(\mu,\lambda) \bf{u}^c,
\end{multline}
for $ i = 1,2,\cdots,n_1^c, j = 1,2,\cdots,n_2^c, k = 1,2,\cdots,n_3^c$ , and 
\begin{multline}\label{fine_scheme_2}
\rho_{i,j,k}^f ({\bf u}_{tt}^f)_{i,j,k} =
 \frac{1}{J_{i,j,k}^f}\Big[D_1^f(N_{11}^fD_1^f{\bf u}_{i,j,k}^f)+D_2^f(N_{22}^fD_2^f{\bf u}_{i,j,k}^f)+\tilde{G}_3^f(N_{33}^f){\bf u}_{i,j,k}^f\\
+D_1^f(N_{12}^fD_2^f{\bf u}_{i,j,k}^f)+D_1^f(N_{13}^fD_3^f{\bf u}_{i,j,k}^f)
+D_2^f(N_{21}^fD_1^f{\bf u}_{i,j,k}^f)+D_2^f(N_{23}^fD_3^f{\bf u}_{i,j,k}^f)\\
+D_3^f(N_{31}^fD_1^f{\bf u}_{i,j,k}^f)
+D_3^f(N_{32}^fD_2^f{\bf u}_{i,j,k}^f)\Big],
\end{multline}
for $ i = 1,2,\cdots,n_1^f, j = 1,2,\cdots,n_2^f, k = 2,\cdots,n_3^f$ and
\begin{multline}\label{fine_scheme_1}
\rho_{i,j,k}^f ({\bf u}_{tt}^f)_{i,j,1} = \frac{1}{J_{i,j,1}^f}\Big[D_1^f(N_{11}^fD_1^f{\bf u}_{i,j,1}^f)+D_2^f(N_{22}^fD_2^f{\bf u}_{i,j,1}^f)+G_3^f(N_{33}^f){\bf u}_{i,j,1}^f\\
+D_1^f(N_{12}^fD_2^f{\bf u}_{i,j,1}^f)+D_1^f(N_{13}^fD_3^f{\bf u}_{i,j,1}^f)+D_2^f(N_{21}^fD_1^f{\bf u}_{i,j,1}^f)+D_2^f(N_{23}^fD_3^f{\bf u}_{i,j,1}^f)\\
+D_3^f(N_{31}^fD_1^f{\bf u}_{i,j,1}^f)+D_3^f(N_{32}^fD_2^f{\bf u}_{i,j,1}^f)+{\bm \eta}_{i,j}\Big]\\
 := M_f(\mu,\lambda){\bf u}^f\big|_\Gamma+\frac{{\bm \eta}_{i,j}}{J^f_{i,j,1}},
\end{multline}
for $ i = 1,2,\cdots,n_1^f, j = 1,2,\cdots,n_2^f$ with
\begin{equation*}
{\bm \eta} = \rho^f\big|_\Gamma\mathcal{P}\Big((\rho^c)^{-1}\tilde{M}_c(\mu,\lambda){\bf u}^c\big|_\Gamma\Big)-M_f(\mu,\lambda){\bf u}^f\big|_\Gamma.
\end{equation*}
For the grids in the fine domain $\Omega^f$  that are lying on the interface $\Gamma$, we impose the continuity of the solution,
\begin{equation}\label{data_continuous_curvi}
{\bf u}_{i,j,1}^f = \mathcal{P}\big({\bf u}_{i,j,n_3^c}^c\big), \ \ \ i = 1,2,\cdots,n_1^c, j = 1,2,\cdots,n_2^c,
\end{equation}
as for the ghost points which is on the direction 3 for the coarse domian, we impose the continuity of the traction force,
\begin{equation}\label{traction_continuous_curvi}
{\bf n}^c\cdot\mathcal{T}^c \big|_\Gamma = \mathcal{R}\left(-{\bf n}^f\cdot\mathcal{T}^f\big|_\Gamma - \frac{h_3^f\omega_1^{(3)}{\bm \eta}}{J^f\big|\nabla r_f^{(3)}\big|}\Big|_\Gamma\right).
\end{equation}

\subsection{Energy estimate}
%We derive an energy estimate, which tells us what interface conditions we shall impose. Also discuss boundary conditions. Be careful with the Jacobian.

%These two sections should not be too long. We shall cite previous works by Petersson and Sjögreen, and also Duru and Virta 2014. 

In this section, we investigate the energy estimate for the semi-discrete form of the SBP scheme in section \ref{sub_section_4_1}. Define the discrete scalar product for the interior of domain by
\begin{equation}\label{scalar_product_discrete_interior}
({\bf u}, {\bf v})_h = h_1h_2h_3\sum_{i=1}^{n_1}\sum_{j=1}^{n_2}\sum_{k=1}^{n_3}\omega_i^{(1)}\omega_j^{(2)}\omega_k^{(3)}J_{i,j,k}{\bf u}^{T}_{i,j,k}{\bf v}_{i,j,k},
\end{equation}
and the scalar product on the interface $\Gamma$,
\begin{equation}\label{scalar_product_discrete_interface}
({\bf u}, {\bf v})_{h,\Gamma} = h_1h_2\sum_{i=1}^{n_1}\sum_{j=1}^{n_2}\omega_i^{(1)}\omega_j^{(2)}J_{i,j,k}\big|\nabla r^{3}\big|{\bf u}_{i,j,k}^{T}{\bf v}_{i,j,k},
\end{equation}
where $k = 1$ for the interface of the fine domain $\Omega_f$ and $k = n_3^c$ for the interface of coarse domain $\Omega_c$ .

Multiplying (\ref{coarse_scheme}) by $h_1^ch_2^ch_3^c\omega_i^{(1)}\omega_j^{(2)}\omega_k^{(3)}J_{i,j,k}^c$ and summing over all grids, we have
\begin{equation}\label{coarse_simple}
({\bf u}_t^c, \rho^c{\bf u}_{tt}^c)_{h^c} = -S_{h^c}({\bf u}_t^c,{\bf u}^c) + B_{h^c}({\bf u}_t^c,{\bf u}^c),
\end{equation}
multiplying (\ref{fine_scheme_1}) by $h_1^fh_2^fh_3^f\omega_i^{(1)}\omega_j^{(2)}\omega_k^{(3)}J_{i,j,k}^f$ and summing over all grids, we obtain
\begin{multline}\label{fine_simple}
({\bf u}_t^f, \rho^f{\bf u}_{tt}^f)_{h^f} = -S_{h^f}({\bf u}_t^f,{\bf u}^f) + B_{h^f}({\bf u}_t^f,{\bf u}^f) \\
+h_3^f\omega_1^{(3)}h_1^fh_2^f\sum_{i=1}^{n_1^f}\sum_{j=1}^{n_2^f}\omega_i^{(1)}\omega_j^{(2)}({\bf u}_t^f)_{i,j,1}^{T}{\bm \eta}_{i,j},
\end{multline}
where  $S_h$ can be found in Appendix ?, and 
\begin{multline*}
B_h({\bf u}_t, {\bf u}) = h_2h_3\sum_{j=1}^{n_2}\sum_{k=1}^{n_3}\omega_j^{(2)}\omega_j^{(3)}\big[({\bf u}_t)_{i,j,k}^{T}\bar{A}_1\bar{\nabla}{\bf u}_{i,j,k}\big]_{i=1}^{i=n_1}\\
+h_1h_3\sum_{i=1}^{n_1}\sum_{k=1}^{n_3}\omega_j^{(1)}\omega_j^{(3)}\big[({\bf u}_t)_{i,j,k}^{T}\bar{A}_2\bar{\nabla}{\bf u}_{i,j,k}\big]_{j=1}^{i=n_2}\\
+h_1h_2\sum_{i=1}^{n_1}\sum_{j=1}^{n_2}\omega_i^{(1)}\omega_j^{(2)}\big[({\bf u}_t)_{i,j,k}^{T}\bar{A}_3\bar{\nabla}{\bf u}_{i,j,k}\big]_{k=1}^{k=n_3},
\end{multline*}
where $\bar{A}_1, \bar{A}_2$ and $\bar{A}_3$ can be found in Appendix ?.  We firstly impose homogeneous Dirichlet boundary condition,
\begin{equation*}
{\bf u}_{i,j,k} = {\bf 0}
\end{equation*}
to the bottom surface ($i = 1,2,\cdots,n_1^c, j = 1,2,\cdots,n_2^c,k = 1$) and a free-surface boundary condition on the top surface ($i = 1,2,\cdots,n_1^f, j = 1,2,\cdots,n_2^f, k = n_3^f$),
\begin{equation*}
\bar{A}_3^f\bar{\nabla} {\bf u}_{i,j,n_3^f}^f = {\bf 0}.
\end{equation*}
As for the direction $1$ and $2$, we assume a periodic boundary condition for simplicity. Then, we have
\begin{equation}\label{boundary_f}
B_{h_f} ({\bf u}_t^f,{\bf u}^f) = -h_1^fh_2^f\sum_{i = 1}^{n_1^f}\sum_{j=1}^{n_2^f}\omega_i^{(1)}\omega_j^{(2)}({\bf u}_t^f)^T_{i,j,1}\bar{A}_3^f\bar{\nabla}_f{\bf u}^f_{i,j,1},
\end{equation}
and 
\begin{equation}\label{bounary_c}
B_{h_c} ({\bf u}_t^c,{\bf u}^c) = h_1^ch_2^c\sum_{i = 1}^{ n_1^c}\sum_{j=1}^{n_2^c}\omega_i^{(1)}\omega_j^{(2)}({\bf u}_t^f)^T_{i,j,n_3^c}\bar{A}_3^c\bar{\nabla}_c{\bf u}^c_{i,j,n_3^c}.
\end{equation}
Finally, we conclude a time derivative of the semi-discrete energy
\begin{multline}\label{semi_energy_1}
\frac{d}{dt}\big[({\bf u}_t^f,\rho^f {\bf u}_t^f)_{h^f} + S_{h^f}({\bf u}^f,{\bf u}^f_t) + ({\bf u}_t^c,\rho^c {\bf u}_t^c)_{h^c} + S_{h^c}({\bf u}^c,{\bf u}^c_t) \big]  = \\
2B_{h^f}({\bf u}_t^f,{\bf u}^f) + 2B_{h^c}({\bf u}_t^c,{\bf u}^c) + 2h_3^f\omega_1^{(3)}h_1^f h_2^f\sum_{i=1}^{n_1^f}\sum_{j=1}^{n_2^f}\omega_i^{(1)}\omega_j^{(2)}({\bf u}_t^f)^T_{i,j,1}{\bm \eta}_{i,j},
\end{multline}
plugging (\ref{boundary_f}) and (\ref{bounary_c}) into (\ref{semi_energy_1}) and combining the definition of the scalar product on the interface $\Gamma$ (\ref{scalar_product_discrete_interface}), we have
\begin{multline}\label{semi_energy_2}
\frac{d}{dt}\left[({\bf u}_t^f,\rho^f {\bf u}_t^f)_{h^f} + S_{h^f}({\bf u}^f,{\bf u}^f_t) + ({\bf u}_t^c,\rho^c {\bf u}_t^c)_{h^c} + S_{h^c}({\bf u}^c,{\bf u}^c_t) \right]  = \\
2\left(-{\bf u}_t^f,\frac{\bar{A}_3^f\bar{\nabla}_f{\bf u}^f}{J^f|\nabla r_f^{(3)}|}\right)_{h_f,\Gamma} + 2\left({\bf u}_t^c,\frac{\bar{A}_3^c\bar{\nabla}_c{\bf u}^c}{J^c|\nabla r_c^{(3)}|}\right)_{h_c,\Gamma} + 2\left({\bf u}_t^f,\frac{h_3^f\omega_1^{(3)}{\bm \eta}}{J^f|\nabla r_f^{(3)}|}\right)_{h_f,\Gamma} = \\
2\left({\bf u}_t^f,-\frac{\bar{A}_3^f\bar{\nabla}_f{\bf u}^f}{J^f|\nabla r_f^{(3)}|} + \frac{h_3^f\omega_1^{(3)}{\bm \eta}}{J^f|\nabla r_f^{(3)}|}\right)_{h_f,\Gamma} +  2\left({\bf u}_t^c,\frac{\bar{A}_3^c\bar{\nabla}_c{\bf u}^c}{J^c|\nabla r_c^{(3)}|}\right)_{h_c,\Gamma} = \\
2\left(\mathcal{P}{\bf u}^c_t,-\frac{\bar{A}_3^f\bar{\nabla}_f{\bf u}^f}{J^f|\nabla r_f^{(3)}|} + \frac{h_3^f\omega_1^{(3)}{\bm \eta}}{J^f|\nabla r_f^{(3)}|}\right)_{h_f,\Gamma} + 2\left({\bf u}_t^c,\frac{\bar{A}_3^c\bar{\nabla}_c{\bf u}^c}{J^c|\nabla r_c^{(3)}|}\right)_{h_c,\Gamma} = \\
2\left({\bf u}_t^c,\mathcal{R}\Big(-\frac{\bar{A}_3^f\bar{\nabla}_f{\bf u}^f}{J^f|\nabla r_f^{(3)}|} + \frac{h_3^f\omega_1^{(3)}{\bm \eta}}{J^f|\nabla r_f^{(3)}|}\Big)\right)_{h_c,\Gamma} + 2\left({\bf u}_t^c,\frac{\bar{A}_3^c\bar{\nabla}_c{\bf u}^c}{J^c|\nabla r_c^{(3)}|}\right)_{h_c,\Gamma}.
\end{multline}
We have (\ref{traction_continuous_curvi}) in the cuvilinear coordinates, then combine with (\ref{bdry3_curvi}) arrives at
\begin{equation*}
\left(\frac{\bar{A}_3^c\bar{\nabla}_c{\bf u}^c}{J^c|\nabla r_c^{(3)}|}\right)_{h_c,\Gamma} = \mathcal{R}\left(\frac{\bar{A}_3^f\bar{\nabla}_f{\bf u}^f}{J^f|\nabla r_f^{(3)}|} - \frac{h_3^f\omega_1^{(3)}{\bm \eta}}{J^f|\nabla r_f^{(3)}|}\right)_{h_c,\Gamma},
\end{equation*}
finally, we conclude that
\begin{equation*}
\frac{d}{dt}\left[({\bf u}_t^f,\rho^f {\bf u}_t^f)_{h^f} + S_{h^f}({\bf u}^f,{\bf u}^f_t) + ({\bf u}_t^c,\rho^c {\bf u}_t^c)_{h^c} + S_{h^c}({\bf u}^c,{\bf u}^c_t) \right]  = 0
\end{equation*}
without external force.

\section{The temporal discretization}
%We present the predictor-corrector discretization in time, and explain how the ghost points are updated. In addtion, we describle the iterative methods. Perhaps we shall also talk about CFL and the time steps. Maybe no fully-discrete energy analysis? That would be very messy. 
The equations are advance in time with an explicit fourth order accurate predictor-corrector time integration method. Like all explicit time stepping methods, there is a maximum time step not exceed CFL stabilitity limit.

In \cite{?}, it is proved that the time step constraint by CFL codition for the Newmark scheme 
\begin{equation*}
\frac{{\bf u}^{n+1}-2{\bf u}^n + {\bf u}^{n-1}}{\Delta_t^2} = {\bf L}_h{\bf u}^n + {\bf F}^n, \ \ \ n = 0,1,\cdots
\end{equation*}
 which is second order is
\begin{equation*}
\frac{\Delta_t^2}{h^2}\kappa_{\text{max}}\leq C_{\text{cfl}}^2.
\end{equation*}
Here, 
$\kappa_{\text{max}}$ is the maximum of the eigenvalue of the matrix 
\[T = \frac{1}{\rho}\left(\begin{array}{ccc}
Tr(N_{11}) &  Tr(N_{12})& Tr(N_{13})\\
Tr(N_{21}) & Tr(N_{22}) & Tr(N_{23})\\
Tr(N_{31}) & Tr(N_{32}) & Tr(N_{33})\end{array}\right). \]
Here, $Tr(N_{ij})$ represents the trace of the matrix $N_{ij},i,j = 1,2,3$. In this paper, we use the predictor-corrector strategy to obtain a fourth order time integrator. In \cite{?}, it shows that the fourth order scheme has a somewhat larger stability limit for the time step, but the way used to approximate eigenvalue is same. We use $C_{\text{cfl}} = $ in the numrical experiments in this paper.

\subsection{Time discretization with SBP scheme}
In the following, we give the detailed procedure about how we apply the fourth order time integrator to the problem (\ref{coarse_scheme}) -- (\ref{traction_continuous_curvi}). 

Let ${\bf u}^{n,c}$ and ${\bf u}^{n,f}$ denote the numerical approximations of ${\bf u}({\bf x},t_n), {\bf x}\in\Omega_c$ and ${\bf u}({\bf x},t_n), {\bf x}\in\Omega_f$ respectively. Here, $t_n = n\Delta t, n = 0,1,\cdots$ and $\Delta_t > 0$ is a constant time step. We present the fourth order time integrator with predictor and corrector in the Algorithm \ref{first_alg}.

\begin{algorithm}
	\caption{Fourth order accurate time steeping for the elastic wave equation with SBP discretization in space}\label{first_alg}
	Given initial conditions ${\bf u}^{0,c}, {\bf u}^{-1,c}$ and ${\bf u}^{0,f}, {\bf u}^{-1,f}$ that satisfies the discretized boundary conditions.
	
	\begin{itemize}
	\item  {Compute the predictor at the interior grid points for both fine and coarse domain
		\begin{equation*}
		   {\bf u}^{c,*,n+1} = 2{\bf u}^{c,n} - {\bf u}^{c,n-1} + \Delta_t^2\rho_c^{-1}
		\end{equation*}
		\begin{equation*}
		{\bf u}^{f,*,n+1} = 2{\bf u}^{f,n} - {\bf u}^{f,n-1} + \Delta_t^2\rho_f^{-1}
		\end{equation*}
	   }
   \item {For the traction free boundary on the top surface of fine domain $\Omega_f$
   }
   \item {For the Dirichlet boundary condition on the bottom surface of coarse domain $\Omega_c$
   }
  \item{Continuity of solution on the interface $\Gamma$
  }
  \item{Continuity of traction force on the interface $\Gamma$
  }
  \item{Evaluate the acceleration at all grids for both coarse and fine domians
  }
  \item{Comoute the corrector at the interior grid points for both coarse and fine domains
  }
 \item {For the traction free boundary on the top surface of fine domain $\Omega_f$
 }
 \item {For the Dirichlet boundary condition on the bottom surface of coarse domain $\Omega_c$
 }
 \item{Continuity of solution on the interface $\Gamma$
 }
 \item{Continuity of traction force on the interface $\Gamma$
 }
	\end{itemize}
\end{algorithm}





\section{Numerical Experiments}

\subsection{Verification of convergence rate}
Show fourth order convergence

\subsection{Gaussian source}
Show that no reflection at the mesh refinement interfaces. 

If the code is incorporated into SW4, it would be nice to solve a practical problem. 

\subsection{Experiment 3}
We shall have an experiment for energy conservation. We also need to evaluate the iterative methods. These can be Experiment 3, or incoporated in the first two experiments.

\section{Conclusion}
\end{document}
