%!TEX root = elastic_3d_sbp.tex
\subsubsection{Boundary forcing}
We start by considering the boundary forcing
\begin{equation}\label{boundary_forcing}
\mathcal{T}\cdot{\bf n}_3 = {\bf g}(x^{(1)},x^{(2)},x^{(3)},t) = (g_1,g_2,g_3)^T,
\end{equation}
where $(x^{(1)},x^{(2)},x^{(3)})$ are on the top surface and ${\bf n}_3$ is the unit outward norm of the top surface. Then in our secheme, the semi-discrtization of (\ref{elastic_equation}) and (\ref{boundary_forcing}) are
\begin{equation}\label{semi_discrete_elastic}
\rho_{\bf i}({\bf u}_{tt})_{\bf i} =\hat{L}_h{\bf u}_{\bf i}, \ \ \ t\geq0 
\end{equation}
and
\begin{equation}\label{discrete_boundary_forcing}
\frac{N_{31} D_1 {\bf u}_{{\bf i}_{\Omega_t}} + N_{32} D_2 {\bf u}_{{\bf i}_{\Omega_t}} + N_{33} \wt{D}_3 {\bf u}_{{\bf i}_{\Omega_t}}}{(J|\nabla r^{(3)}|)_{{\bf i}_{\Omega_t}}}  = {\bf g}_{i_1,i_2}, \ \ \ t\geq0.
\end{equation}
Here, $D_{1,2}$ are fourth order central difference operators for the first derivative $\partial_{1,2}$ witout using any ghost points and $\wt{D}_3$ is a fourth order SBP operator as in (\ref{sbp_1st_1}) with ghost points. Multiplying (\ref{semi_discrete_elastic}) by $h_1h_2h_3\omega_{i_1}^{(1)}\omega_{i_2}^{(2)}\omega_{i_3}^{(3)}J_{\bf i}$ and summing over all grids, we have
\begin{multline*}
({\bf u}_t,\rho{\bf u}_{tt})_h = -S_h({\bf u}_t,{\bf u}) + h_1h_2\sum_{i_1=1}^{n_1}\sum_{i_2=1}^{n_2}\omega_{i_1}^{(1)}\omega_{i_2}^{(2)}({\bf u}_t)_{{\bf i}_{\Omega_t}}^T(N_{31}D_1{\bf u}_{{\bf i}_{\Omega_t}} + N_{32}D_2{\bf u}_{{\bf i}_{\Omega_t}} +N_{33}\wt{D}_3{\bf u}_{{\bf i}_{\Omega_t}}),
\end{multline*}
where the bilinear form $S_h(\cdot,\cdot)$ is symmetric and positive semi-definite. And the equation can be rewritten as
\begin{multline}\label{time_derivative_energy}
({\bf u}_t,\rho{\bf u}_{tt})_h + S_h({\bf u}_t,{\bf u}) = 
h_1h_2\sum_{i_1=1}^{n_1}\sum_{i_2=1}^{n_2}\omega_{i_1}^{(1)}\omega_{i_2}^{(2)}({\bf u}_t)_{{\bf i}_{\Omega_t}}^T(N_{31}D_1{\bf u}_{{\bf i}_{\Omega_t}} + N_{32}D_2{\bf u}_{{\bf i}_{\Omega_t}} +N_{33}\wt{D}_3{\bf u}_{{\bf i}_{\Omega_t}}).
\end{multline}
We define the discrete energy as
\begin{equation*}
E_h := ({\bf u}_t, \rho{\bf u}_t)_h + S_h({\bf u},{\bf u}),
\end{equation*}
which is an analog of the continuous energy. 
From (\ref{time_derivative_energy}), we have
\begin{equation}\label{time_derivative_energy_2}
\frac{dE_h}{dt} = 2h_1h_2\sum_{{i_1}=1}^{n_1}\sum_{{i_2}=1}^{n_2}\omega_{i_1}^{(1)}\omega_{i_2}^{(2)}({\bf u}_t)_{{\bf i}_{\Omega_t}}^T(N_{31}D_1{\bf u}_{{\bf i}_{\Omega_t}} + N_{32}D_2{\bf u}_{{\bf i}_{\Omega_t}} +N_{33}\wt{D}_3{\bf u}_{{\bf i}_{\Omega_t}}).
\end{equation}
To obtain energy stability, we need to impose boundary forcing condition such that the right hand side of (\ref{time_derivative_energy_2}) is non-positive. Here, we use the ghost points in the direction $3$ as the additional degrees of freedom to impose the boundary forcing condition. From (\ref{discrete_boundary_forcing}), we obtain
\begin{equation}\label{eq1}
(J|\nabla r^{(3)}|)_{{\bf i}_{\Omega_t}} \left(\begin{array}{c}
(g_1)_{i_1,i_2} \\
(g_2)_{i_1,i_2}\\
(g_3)_{i_1,i_2}\end{array}\right) = \left(\begin{array}{c}
\sum_{m=1}^2\sum_{k=1}^3(N_{3m})_{1k}D_m(u_k)_{{\bf i}_{\Omega_t}} +\sum_{k=1}^3(N_{33})_{1k}\wt{D}_3(u_k)_{{\bf i}_{\Omega_t}}\\
\sum_{m=1}^2\sum_{k=1}^3(N_{3m})_{2k}D_m(u_k)_{{\bf i}_{\Omega_t}} 
+\sum_{k=1}^3(N_{33})_{2k}\wt{D}_3(u_k)_{{\bf i}_{\Omega_t}} \\
\sum_{m=1}^2\sum_{k=1}^3(N_{3m})_{3k}D_m(u_k)_{{\bf i}_{\Omega_t}}
+\sum_{k=1}^3(N_{33})_{3k}\wt{D}_3(u_k)_{{\bf i}_{\Omega_t}}
\end{array}\right),
\end{equation}
where the subscript $jk$ in $(N_{3m})_{jk}, m = 1,2,3, j,k = 1,2,3$ represents the $j^{th}$ row and $k^{th}$ column of the matrix $N_{3m}$. Then for any fixed points ${\bf i}_{\Omega_t} = (i_1,i_2,n_3)$, based on (\ref{sbp_1st_1}),
\begin{align}\label{eq2}
h_3 \tilde{D}_3(u_k)_{i_1,i_2,n_3} = -\tilde{d}_0 (u_k)_{i_1,i_2,n_3+1} - \sum_{j = 1}^4  \tilde{d}_k (u_k)_{i_1,i_2,n_3+1-k},
\end{align}
for $k = 1,2,3$. Plug (\ref{eq2}) into (\ref{eq1}), we get
\begin{align}\label{matrix_boundary}
&\hspace{0.4cm}\frac{\tilde{d}_0}{h_3}\left(\begin{array}{ccc}
(N_{33})_{11}&(N_{33})_{12} & (N_{33})_{13}\\
(N_{33})_{21} &(N_{33})_{22} &(N_{33})_{23}\\
(N_{33})_{31} &(N_{33})_{32} &(N_{33})_{33}
\end{array}\right) \left(\begin{array}{c}
(u_1)_{i_1,i_2,n_3+1}\\
(u_2)_{i_1,i_2,n_3+1}\\
(u_3)_{i_1,i_2,n_3+1}
\end{array}\right)\nonumber\\
&= \left(\begin{array}{ccc}
\sum_{m=1}^3\frac{1}{h_3}(N_{33})_{1m}\sum_{k=1}^4\tilde{d}_k(u_m)_{i_1,i_2,n_3+1-k}\\
\sum_{m=2}^3\frac{1}{h_3}(N_{33})_{2m}\sum_{k=1}^4\tilde{d}_k(u_m)_{i_1,i_2,n_3+1-k}\\
\sum_{m=2}^3\frac{1}{h_3}(N_{33})_{3m}\sum_{k=1}^4\tilde{d}_k(u_m)_{i_1,i_2,n_3+1-k}
\end{array}\right)\nonumber
+\left(\begin{array}{ccc}
\sum_{m=1}^2\sum_{k=1}^3(N_{3m})D_m(u_k)_{i_1,i_2,n_3}\\
\sum_{m=1}^2\sum_{k=1}^3(N_{3m})D_m(u_k)_{i_1,i_2,n_3}\\
\sum_{m=1}^2\sum_{k=1}^3(N_{3m})D_m(u_k)_{i_1,i_2,n_3}
\end{array}\right)\nonumber\\
&\hspace{0.4cm}-\left(\begin{array}{ccc}
(J|\nabla r^{(3)}|)_{i_1,i_2,n_3}(g_1)_{i_1,i_2}\\
(J|\nabla r^{(3)}|)_{i_1,i_2,n_3}(g_2)_{i_1,i_2}\\
(J|\nabla r^{(3)}|)_{i_1,i_2,n_3}(g_3)_{i_1,i_2}
\end{array}\right).
\end{align}
We then have the value for ghost points ${\bf u}_{i_1,i_2,n_3+1}$  after solving the linear system $(\ref{matrix_boundary})$ which is compatible with the discrete boundary focing condition (\ref{discrete_boundary_forcing}). Thus, the resulting scheme is energy conservative $\frac{dE_h}{dt} = 0$ if ${\bf g} = {\bf 0}$.
