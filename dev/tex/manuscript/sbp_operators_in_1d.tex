%!TEX root = SISC_elastic_3d.tex
\subsection{SBP operators in $1$D}\label{sec_sbp_1d}
Consider a uniform discretization of the domain $x\in[0,1]$ with the grids,
\[\wt
	{\bf x} = [x_0,x_1,\cdots,x_n,x_{n+1}]^T,\ \  x_i = (i-1)h,\ \ i = 0,1,\cdots,n,n+1,\ \ h = 1/(n-1),\]
where $i = 1,n$ correspond to the grid points at the boundary, and $i = 0,n+1$ are ghost points outside of the physical domain. The  operator $D \approx \frac{\partial }{\partial x}$ is a first derivative SBP operator \cite{Kreiss1974,Strand1994} if 
\begin{equation}\label{first_sbp}
({\bf u}, D{\bf v})_h = -(D{\bf u},{\bf v})_h - u_1v_1 + u_nv_n,
\end{equation}
with a scalar product
\begin{equation}\label{inner_product}
({\bf u},{\bf v})_h = h\sum_{i = 1}^{n}\omega_iu_iv_i.
\end{equation}
Here, $0<\omega_i < \infty $ are the weights of scalar product. The SBP operator $D$ has a centered difference stencil at the grid points away from the boundary and the corresponding weights $\omega_i = 1$. To satisfy the SBP identity (\ref{first_sbp}), the coefficients in $D$ are  modified at a few points near the boundary and the corresponding weights $\omega_i \neq 1$. The operator $D$ does not use any ghost points. To discretize the elastic wave equation, we also need to approximate the second derivative with a variable coefficient $(\gamma(x)u_x)_x$. Here, the known function $\gamma(x)>0$ describes the property of the material. There are two different fourth order accurate SBP operators for the approximation of $(\gamma(x)u_x)_x$. The first one $\wt{G}(\gamma){\bf u} \approx (\gamma(x)u_x)_x $, derived by Sj\"ogreen and Petersson \cite{sjogreen2012fourth}, uses one ghost point outside each boundary, and satisfies the second derivative SBP identity,
\begin{equation}\label{sbp_2nd_1}
({\bf u}, \wt{G}(\gamma){\bf v})_h = -S_\gamma({\bf u},{\bf v})-u_1\gamma_1\wt{\bf b}_1{\bf v} + u_n\gamma_n\wt{\bf b}_n {\bf v}.
\end{equation}
Here, the bilinear form $S_\gamma({\bf u},{\bf v}) = (D{\bf u},\gamma D{\bf v})_h + ({\bf u}, P(\gamma){\bf v})_{hr}$ without using any ghost points, $(\cdot,\cdot)_{hr}$ is a standard discrete scalar $L^2$ inner product. The positive semi-definite operator $P(\gamma)$ is small for smooth grid functions and non-zero for odd-even modes , see \cite{petersson2015wave,sjogreen2012fourth} for details, and $S_\gamma(\cdot,\cdot)$ is symmetric positive semi-definite. The operators $\wt{\bf b}_1$ and $\wt{\bf b}_n$ take the form %Using the right boundary as an example, we have 
\begin{equation}\label{sbp_1st_1}
\wt{\bf b}_1 {\bf v} = \frac{1}{h}\sum_{i=0}^{4} \wt{d}_i v_i, \quad\wt{\bf b}_n {\bf v} = \frac{1}{h}\sum_{i=n-3}^{n+1} \wt{d}_i v_i.
\end{equation}
They are fourth order approximations of the first derivative $v_x$ on the left and right boundaries, respectively. We note that the notation $\wt{G}(\gamma){\bf v}$ implies that the operator $\wt{G}$ uses ${\bf v}$ on all grid points $\wt{\bf x}$, but $\wt{G}(\gamma){\bf v}$ only returns values on the grid ${\bf x}$ without ghost points. Therefore, when writing in matrix form, $\wt{G}$ is a rectangular matrix of size $n$ by $n+2$.

 In \cite{wang2018fourth}, a method was developed to convert the SBP operator $\wt{G}(\gamma)$ to another SBP operator $G(\gamma)$ which does not use any ghost point and satisfy
 \begin{equation}\label{sbp_2nd_2}
 ({\bf u}, G(\gamma){\bf v})_h = -S_\gamma({\bf u},{\bf v})-u_1\gamma_1{\bf b}_1{\bf v} + u_n\gamma_n{\bf b}_n{\bf v},
 \end{equation}
 where $S_\gamma(\cdot,\cdot)$ is symmetric positive semi-definite. 
 Here, ${\bf b}_1$ and ${\bf b}_n$ are also finite difference operators for the first derivative at the boundaries, and are constructed to fourth order accuracy. They take the form
 \begin{equation}\label{sbp_1st_2}
 {\bf b}_1 {\bf v} = \frac{1}{h}\sum_{i=1}^{5} d_i v_i,\quad {\bf b}_n {\bf v} = \frac{1}{h}\sum_{i=n-4}^{n} d_i v_i.
 \end{equation}
   In this case, ${G}(\gamma)$ is square in matrix form. We note that in  \cite{Mattsson2012}, Mattsson constructed a similar SBP operator with a third order approximation of the first derivative at the boundaries.  % We refer \cite{mattsson2004summation} for another SBP operator which is used to approximate $(\gamma(x)u_x)_x $ without using any ghost points.
 
For the second derivative SBP operators $\wt{G}(\gamma)$ in (\ref{sbp_2nd_1}) and $G(\gamma)$ in (\ref{sbp_2nd_2}), both of them use a fourth order five points centered difference stencil to approximate $(\gamma u_x)_x$ at the interior points away from the boundaries. For the first and the last six grid points close to the boundaries, the operators $G(\gamma)$ and $\wt{G}(\gamma)$ use second order accurate one-sided difference stencils. They are designed to satisfy (\ref{sbp_2nd_2}) and (\ref{sbp_2nd_1}), respectively.

In the following section, we use a combination of two SBP operators, $\wt{G}(\gamma)$ and $G(\gamma)$, to develop a multi-block finite difference discretization for the elastic wave equation. The first SBP operator is $\wt{G}(\gamma)$ with ghost point, and the seond SBP operator $G(\gamma)$, converted from $\wt{G}(\gamma)$, does not use ghost point.
