%!TEX root = elastic_3d_sbp.tex
\subsection{Method of manufactured solutions}\label{manufactured_sol}
We take the computation domain to be 
\begin{equation}\label{coarse_domain_manufactured}
\left\{
\begin{aligned}
& x^{c,1} = 2\pi r^1\\
& x^{c,2} = 2\pi r^2\\
& x^{c,3} = r^3\theta_i\big(r^1,r^2\big) + (1-r^3)\theta_b\big(r^1,r^2\big)
\end{aligned}
\right.
\end{equation}
for coarse domain $\Omega^c$. Here, $0\leq r^1, r^2, r^3\leq 1$, $f_i$ is the interface surface geometry,
\begin{equation}\label{iterface_geometry}
\theta_i\big(r^1,r^2\big) = \pi+0.2\sin(4\pi r^1)+0.2\cos(4\pi r^2),
\end{equation}
and 
$\theta_b$ is the bottom surface geometry,
\begin{equation}\label{bottom_geometry}
\theta_b\big(r^1,r^2\big) = 0.2\exp\left(-\frac{(r^1-0.6)^2}{0.04}\right)+0.2\exp\left(-\frac{(r^2-0.6)^2}{0.04}\right).
\end{equation}
As for the fine domian $\Omega^f$, it is chosen to be
\begin{equation}\label{fine_domain_manufactured}
\left\{
\begin{aligned}
& x^{f,1} = 2\pi r^1\\
& x^{f,2} = 2\pi r^2\\
& x^{f,3} = r^3\theta_t\big(r^1,r^2\big) + (1-r^3)\theta_i\big(r^1,r^2\big),
\end{aligned}
\right.
\end{equation}
where $0\leq r^1, r^2, r^3\leq 1$, $\theta_t$ is the top surface geometry,
\begin{equation}\label{top_geometry}
\theta_t\big(r^1,r^2\big) = 0.2\exp\left(-\frac{(r^1-0.5)^2}{0.04}\right)+0.2\exp\left(-\frac{(r^2-0.5)^2}{0.04}\right),
\end{equation}
and $\theta_i$ is the interface geometry which is given in (\ref{iterface_geometry}). Note that the subdomian 
$\Omega^f$ is on the top of the subdomain $\Omega^c$. For both fine and coarse domians, let the density vary according to
\begin{equation}\label{density_function}
\rho(x^1,x^2,x^3) = 2 + \sin(x^1+0.3)\sin(x^2+0.3)\sin(x^3-0.2),
\end{equation}
and material parameters $\mu, \lambda$ satisfy
\begin{equation}\label{mu_function}
\mu(x^1,x^2,x^3) = 3 + \sin(3x^1+0.1)\sin(3x^2+0.1)\sin(x^3),
\end{equation}
and 
\begin{equation}\label{lambda_function}
\lambda(x^1,x^2,x^3)  = 21+ \cos(x^1+0.1)\cos(x^2+0.1)\sin^2(3x^3),
\end{equation}
respectively. Besides, we impose a boundary forcing on the top surface and Dirichlet boundary conditions for the other boundaries. The external forcing, top boundary forcing ${\bf g}$ and initial conditions are chosen such that ${\bf u}(\cdot,t) = (u_1(\cdot,t),u_2(\cdot,t),u_3(\cdot,t))^T$ with
\begin{align*}
u_1(\cdot,t) &= \cos(x^1+0.3)\sin(x^2+0.3)\sin(x^3+0.2)\cos(t^2),\\
u_2(\cdot,t) &= \sin(x^1+0.3)\cos(x^2+0.3)\sin(x^3+0.2)\cos(t^2),\\
u_3(\cdot,t) &= \sin(x^1+0.2)\sin(x^2+0.2)\cos(x^3+0.2)\sin(t).
\end{align*}
For example, for the boundary forcing on the top surface, we impose 
\begin{equation}\label{traction_force}
{\bf g} = (g_1,g_2,g_3)^T = \sum_{i=1}^3\left(\sum_{j = 1}^3 M_{ij}^f\frac{\partial{\bf u}}{\partial x^{(j)}}\right) n^{f,+,i}_3,
\end{equation}
where, $M_{ij}^f$ and $n^{f,+,i}_3 $ can be found in (\ref{M_definition}) and (\ref{outward_normal}) respectively.